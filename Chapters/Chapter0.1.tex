\chapter*{Abstract}

Currently, many of the software solutions that can be found in the market have some degree of processing in the cloud, thus providing hardware and software on demand, bringing to the developer tools that 10 years ago were unthinkable but today can be reached with a single click.

The virtues of cloud computing, although broad, imply high operating costs and in some cases limit the access not necessarily because of budget, but by the very nature of the technology involved in communications (via the Internet), which adds operational difficulties in those applications where the response time is a key factor. In these cases, cloud computing cannot be even taken as an option.

In this work, we will use edge computing and hardware acceleration using FPGAs as a tool that allows optimizing the bandwidth and data rate sent to the cloud for further processing, with the aim of increasing the number of research projects in the field of Internet of Things in the country.
