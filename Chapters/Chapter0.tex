\chapter*{Resumen}

Muchas de las soluciones de software que se pueden encontrar en el mercado actualmente tienen algún grado de procesamiento en la nube, lo que proporciona hardware y software bajo demanda, brindando así al desarrollador herramientas que hace 10 años eran impensables y hoy están al alcance de un click.

Las virtudes de la computación en la nube, aunque amplias, implican altos costos operativos y en algunos casos limitan el acceso no por su presupuesto sino por la propia naturaleza de la tecnología involucrada en las comunicaciones (vía Internet), que adiciona dificultades de tipo operativas en aquellas aplicaciones en las que el tiempo de respuesta es un factor clave. En estos casos, la computación en la nube no se puede tomar ni siquiera como una opción.

En este trabajo se pretende hacer uso de la computación de borde y la aceleración por hardware usando FPGAs como una herramienta que optimiza el ancho de banda y la cuota de datos enviados a la nube para su posterior procesamiento, con el objetivo de incrementar el número de investigaciones y proyectos de investigación en el ramo del Internet de las Cosas en el país.
