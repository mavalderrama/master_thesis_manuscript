% Chapter 1

\chapter{Introducción} % Write in your own chapter title
\label{Chapter1}
\lhead{Capítulo 1. \emph{Introducción}} % Write in your own chapter title to set the page header

La computación en la nube como medio de procesamiento y análisis se convirtió en una herramienta con recursos casi ilimitados para los desarrolladores desde su masificación hace unos pocos años. La computación en la nube cuenta con múltiples bondades; entre ellas encontramos el Internet de las Cosas (\textit{IoT}) como una de las aplicaciones más importantes dada la gran cantidad de datos y recursos necesarios para procesar los altos volúmenes de información generados por las redes de sensores a los que usualmente se encuentra acoplada. Hoy, encontramos el término computación de borde, que aunque es un esquema relativamente nuevo de computación, ha llamado la atención de múltiples investigadores dado que implica llevar parte de los servicios de la nube al lugar donde realmente se producen los datos \cite{bonomi2014fog}. Realizar parte del procesamiento en el borde le permite al desarrollador acceder a aplicaciones sensibles a la latencia incapaces de operar directamente sobre la nube, disminución de los costos de analítica, reducciones en el almancenamiento y volumen de datos procesados al interior de la nube, entre otros.

%tek

Aparte de las diferentes virtudes que ofrece la computación de borde, también encontramos marcadas limitaciones como la reducida potencia de cálculo de algunos de los dispositivos. En este sentido, la aceleración por hardware y la reconfiguración parcial dinámica, juegan un rol importante, aportando potencia de cálculo en un espacio reducido en función de las necesidades de procesamiento requeridas por la aplicación.

El desarrollo de este documento, resultado del trabajo alrededor de la temática anteriormente enunciada, tiene la siguiente estructura:

\begin{itemize}
    \item El capítulo 1 contiene la introducción al documento.
    \item En el capítulo 2 se explica la definición del problema al que este documento quiere abordar y dar solución.
    \item El capítulo 3 enumera los diferentes objetivos que se pretenden cumplir al final del proyecto.
    \item El capítulo número 4 explica la justificación de la necesidad de realizar y llevar a cabo el proyecto.
    \item En el capítulo número 5 se exponen los alcances del proyecto a nivel de tareas a realizar basado en los diferentes objetivos.
    \item Todo el capítulo 6 expone el marco teórico del tema general que aborda todo el proyecto.
    \item En el capítulo número 7 se discriminan los recursos necesarios para llevar a cabo el proyecto a buen término.
    \item El capítulo 8 expone la metodología usada en el proyecto.
    \item En el capítulo número 9 se exponen los criterios de elección para la plataforma de desarrollo.
    \item El diseño de software y hardware se trata en el capítulo 10; aquí se muestra sistemáticamente el proceso de diseño para ambas partes del proyecto.
    \item El capítulo número 11 trata el diseño experimental; en este se tratan las diferentes actividades para realizar las pruebas con las que se obtendrán los diferentes resultados.
    \item En el capítulo número 12 se expone comparativamente la red neuronal software usando Matlab y la red neuronal implementada en hardware con el objetivo de validar el modelo final implementado.
    \item El análisis de los resultados de la implementación se trata en el capítulo número 13.
    \item Finalmente, las conclusiones se exponen en el capítulo número 14.
\end{itemize}
